% create a basic latex article

\documentclass[12pt]{article}
% Add library to import python code from a file and shwo it 

\usepackage{listings}
\usepackage{amsmath}
\usepackage{amssymb}
\usepackage{graphicx}

\title{Seminar - Markowitz Portfolio Optimization}
\author{PUJOL Martin \\ RAMPONT Martin \\ THOMASSIN Pablo \\ STRIEBIG Maximilien}

\date{\today}


\usepackage{amsmath}

\begin{document}

\maketitle

\section*{Theoretical Part}

\subsection*{Problem Description}

In the realm of financial portfolio management, the Markowitz portfolio optimization problem is a classical and essential topic. The primary objective is to allocate weights to different assets in a portfolio to maximize the expected return while minimizing the overall portfolio risk. Let's consider a portfolio with $n$ assets. The goal is to find the optimal set of weights for these assets.

\subsubsection*{Formalization}

An other formulation of the problem is to minimize the portfolio risk $\sigma_p$ while achieving a target expected return $\mu$:

The objective is to find the vector of weights $\mathbf{w} = [w_1, w_2, \ldots, w_n]$ that minimizes the portfolio risk $\sigma_p$ while achieving a given expected portfolio return $\mu$:

\begin{equation}
    \begin{aligned}
        \text{Minimize} \quad   & \sigma_p = \sqrt{\sum_{i=1}^{n}\sum_{j=1}^{n} w_i w_j \sigma_i \sigma_j \rho_{ij}} \quad \text{(Portfolio risk)} \\
        \text{Subject to} \quad & \mu = \sum_{i=1}^{n} r_i w_i \quad \text{(Expected portfolio return)}                                             \\
                                & \sum_{i=1}^{n} w_i = 1 \quad \text{(Sum of weights equals 1)}                                                        \\
                                & w_i \geq 0 \quad \text{(Non-negativity constraint)}
    \end{aligned}
\end{equation}


In our problem we only want to minimize the portfolio risk $\sigma_p$.



After modifying the objective function to be unconstrained, we obtain the following problem formulation:

\begin{equation}
    \begin{aligned}
        \text{Minimize} \quad   & \sigma_p^2 = \frac{1}{(\sum_{k=1}^{n} e^{x_k})^{2}} \sum_{i=1}^{n}\sum_{j=1}^{n} e^{x_i} e^{x_j}\sigma_i \sigma_j \rho_{ij} \quad \text{(Portfolio risk)} \\
    \end{aligned}
\end{equation}

With the variable change:

\begin{equation}
    w_i = \frac{e^{x_i}}{\sum_{j=1}^{n} e^{x_j}}
\end{equation}

We also calculate the derivate of the objective function:

\begin{equation}
    \begin{aligned}
        \frac{\partial}{\partial x_n} \sigma_p^2 & = \frac{-2 e^{x_n}}{(\sum_{k=1}^{N} e^{x_k})^{3}} \sum_{i=1}^{N}\sum_{j=1}^{N} e^{x_i} e^{x_j}\sigma_i \sigma_j \rho_{ij} + 2 \sigma_n e^{x_n} \sum_{j=1}^{N} e^{x_j}\sigma_j \rho_{nj} \\
    \end{aligned}
\end{equation}

\section*{Numerical Part}

\subsection*{Selected Optimization Methods}

We have chosen to implement two different numerical optimization methods to solve the Markowitz portfolio optimization problem:


\begin{enumerate}
    \item Method 1: [Fixed step : Gradient descent]
    \item Method 2: [Variable step : Golden section search]
\end{enumerate}

\subsection*{Algorithm Implementation}

Below are the basic functions describing the two chosen algorithms:

\subsubsection*{Method 1: [Fixed step : Gradient descent]}

% Insert code or pseudocode for Method 1 implementation 

The gradient step method can be formulated as follows:

\begin{equation}
    x_{k+1} = x_k - \alpha \nabla f(x_k)
\end{equation}
    
Where $\alpha_k$ is the step size and $\nabla f(x_k)$ is the gradient of the objective function at $x_k$.

We implemented in the following python code :

% Insert the funtion descente_markowitz from resolution_methods/fixed_step_descent.py

\begin{lstlisting}[language=Python]
    def descente_markowitz(x0,covariance_matrix,e,p, max_iter=10000):
        """ 
        Gradient descent algorithm to minimize the portfolio risk.

        Parameters:
            x0 (numpy array): Initial portfolio weights.
            covariance_matrix (numpy array): Covariance matrix of asset returns.
            e (float): Tolerance for the stopping criterion.
            p (float): Step size.
            max_iter (int): Maximum number of iterations.

        Returns:
            numpy array: Optimal portfolio weights.
            int: Number of iterations.
            float: Portfolio risk.
        """
        k = 0
        ek = 2*e
        while ek>=e and k<max_iter:
            wk = -1*grad_f(x0,covariance_matrix)
            x0 = x0+p*wk
            ek = np.linalg.norm(p*wk)
            k+=1
        return np.exp(x0)/np.sum(np.exp(x0)),k,f(x0,covariance_matrix)
\end{lstlisting}



\subsubsection*{Method 2: [Variable step : Golden section search]}

The golden section search method can be formulated as follows:

\begin{equation}
    x_{k+1} = x_k + \alpha_k d_k
\end{equation}
    
    Where $\alpha_k$ is the step size and $d_k$ is the search direction.

% Explain how alpha_k is calculated in golden section search

The golden section determine the optimal step size $\alpha_k$ by minimizing the objective function along the search direction $d_k$.


We implemented in the following python code :

% Insert the funtion descente_markowitz from the file resolution_methods/descente2.py

\begin{lstlisting}[language=Python]
    def descente2(x0,covariance_matrix,e,method='golden_section', max_iter=10000):
        """
        Golden section algorithm to minimize the portfolio risk.

        Parameters:
            x0 (numpy array): Initial portfolio weights.
            covariance_matrix (numpy array): Covariance matrix of asset returns.
            e (float): Tolerance for the stopping criterion.
            method (str): Line search method.
            max_iter (int): Maximum number of iterations.
            
        Returns:
            numpy array: Optimal portfolio weights.
            int: Number of iterations.
            float: Portfolio risk.
        """
        k = 0
        ek = 2*e
        while ek>=e and k<max_iter:
            wk = -1*grad_f(x0,covariance_matrix)
            p = pk(x0,covariance_matrix,wk,method)
            x0 = (x0+p*wk)
            ek = np.linalg.norm(p*wk)
            k+=1
        return np.exp(x0)/np.sum(np.exp(x0)),k,f(x0,covariance_matrix)

        def pk(x,covariance_matrix, wk, method='golden_section'):
        """
        Line search to find the optimal step size.
    
        Parameters:
            x (numpy array): Portfolio weights.
            covariance_matrix (numpy array): Covariance matrix of asset returns.
            wk (numpy array): Negative gradient of the portfolio risk.
            method (str): Line search method.
    
        Returns:
            float: Optimal step size.
        """
    
    
        if method == 'golden_section':
    
            alpha = golden_section_line_search(x,covariance_matrix, wk)
    
        else:
    
            alpha = wolfe_conditions_line_search(x, wk)
    
        return alpha
    
    def golden_section_line_search(x,covariance_matrix, wk, c1=1e-4, max_iter=100):
        """
        Golden section line search to find the optimal step size.
        
        Parameters:
            x (numpy array): Portfolio weights.
            covariance_matrix (numpy array): Covariance matrix of asset returns.
            wk (numpy array): Negative gradient of the portfolio risk.
            c1 (float): Parameter for the stopping criterion.
            max_iter (int): Maximum number of iterations.
            
            Returns:
            float: Optimal step size.
            """
        a = 0.0
        b = 1.0
        tau = 0.618
    
        for _ in range(max_iter):
            alpha1 = a + (1 - tau) * (b - a)
            alpha2 = a + tau * (b - a)
    
            f1 = f(x+ alpha1* wk,covariance_matrix)
            f2 = f(x +alpha2* wk,covariance_matrix)
    
            if f2 > f1:
                b = alpha2
            else:
                a = alpha1
    
            if abs(alpha2 - alpha1) < c1:
                break
    
        return (alpha1 + alpha2) / 2.0

\end{lstlisting}

\subsection*{Results}

\subsubsection*{Data}

Using Yahoo Finance, we have downloaded the daily adjusted closing prices of the following assets and then obtain the following covariance matrix:

\begin{table}[h]
    \centering
    \begin{tabular}{l|ccccc}
          & AAPL   & AMZN   & GOOGL  & MSFT   & TSLA   \\
    \hline
    AAPL  & 0.000341 & 0.000311 & 0.000289 & 0.000267 & 0.000407 \\
    AMZN  & 0.000311 & 0.000730 & 0.000407 & 0.000364 & 0.000523 \\
    GOOGL & 0.000289 & 0.000407 & 0.000489 & 0.000316 & 0.000400 \\
    MSFT  & 0.000267 & 0.000364 & 0.000316 & 0.000381 & 0.000353 \\
    TSLA  & 0.000407 & 0.000523 & 0.000400 & 0.000353 & 0.001453 \\
    \end{tabular}
    \caption{Covariance Matrix}
    \label{tab:covariance_matrix}
\end{table}



\subsubsection*{Research and results}

\subsubsection*{Research and results}

The initial portfolio risk was calculated to be 0.021505979626867473. The initial weights of the assets in the portfolio were as follows:

\begin{itemize}
    \item AAPL: 0.19058765
    \item AMZN: 0.14685853
    \item GOOGL: 0.15496298
    \item MSFT: 0.22780396
    \item TSLA: 0.27978688
\end{itemize}

After running the fixed step descent method for 10000 iterations, the final portfolio risk was reduced to 0.02079916040948402. The final weights of the assets in the portfolio were:

\begin{itemize}
    \item AAPL: 0.19819201
    \item AMZN: 0.190089
    \item GOOGL: 0.19158406
    \item MSFT: 0.20515562
    \item TSLA: 0.2149793
\end{itemize}

Similarly, after running the backtracking method for 10000 iterations, the final portfolio risk was reduced to 0.02081173878844736. The final weights of the assets in the portfolio were:

\begin{itemize}
    \item AAPL: 0.19793261
    \item AMZN: 0.18951809
    \item GOOGL: 0.19105297
    \item MSFT: 0.2053556
    \item TSLA: 0.21614072
\end{itemize}

\subsubsection*{Interpretation}


\subsubsection*{Interpretation}

Both the fixed step and backtracking methods were used to optimize the portfolio weights with the goal of minimizing portfolio risk. The initial portfolio risk was 0.021505979626867473. 

After 10000 iterations, the fixed step method reduced the portfolio risk to 0.02079916040948402, while the backtracking method reduced it to 0.02081173878844736. The results are very close, indicating that both methods are effective at reducing portfolio risk. 

However, the fixed step method resulted in a slightly lower portfolio risk than the backtracking method. This suggests that the fixed step method may be slightly more effective at optimizing portfolio weights to minimize risk. 

It is also important to note that the variable step method required more iterations to reach the final portfolio risk. This is because the step size is calculated at each iteration, which requires more computation. 

The fact that the variable step is supposed to be more efficient than the fixed step method is not reflected in the results. This may be due to the fact that the fixed step method is already very efficient at reducing the portfolio risk and the golden section search need more iterations to converge.

In terms of the final portfolio weights, both methods resulted in similar weights for each asset. This indicates that both methods are consistent in their optimization process. 

It's important to note that the effectiveness of these methods can depend on the specific characteristics of the portfolio and the market conditions. Therefore, it's recommended to use a combination of methods and to continually monitor and adjust the portfolio weights as necessary.



\section*{Annexe : Objective function and constraints}

\subsection*{Objective function}

For solving the Markowitz portfolio optimization problem, we have chosen two numerical optimization methods:

% I want to use non-constraint optimization methods 
% Fist let's forget about the Expected return constraint

I order to simplify the problem, we will first forget about the expected return constraint. We will only focus on minimizing the portfolio risk $\sigma_p$.

\begin{equation}
    \begin{aligned}
        \text{Minimize} \quad   & \sigma_p = \sqrt{\sum_{i=1}^{n}\sum_{j=1}^{n} w_i w_j \sigma_i \sigma_j \rho_{ij}} \quad \text{(Portfolio risk)} \\
        \text{Subject to} \quad & \sum_{i=1}^{n} w_i = 1 \quad \text{(Sum of weights equals 1)}                                                        \\
                                & w_i \geq 0 \quad \text{(Non-negativity constraint)}
    \end{aligned}
\end{equation}

% To restruct the weight vector to be positive, we can use a variable change
% w_i = exp(x_i) with x_i \in R

To restruct the weight vector to be positive, we can use a variable change:
\begin{equation}
    w_i = e^{x_i} \quad \text{with} \quad x_i \in \mathbb{R}
\end{equation}

The problem becomes:

\begin{equation}
    \begin{aligned}
        \text{Minimize} \quad   & \sigma_p = \sqrt{\sum_{i=1}^{n}\sum_{j=1}^{n} e^{x_i} e^{x_j} \sigma_i \sigma_j \rho_{ij}} \quad \text{(Portfolio risk)} \\
        \text{Subject to} \quad & \sum_{i=1}^{n} e^{x_i} = 1 \quad \text{(Sum of weights equals 1)}    
    \end{aligned}
\end{equation}

We can also forget about the square root in the objective function, as it does not change the optimal solution.

\begin{equation}
    \begin{aligned}
        \text{Minimize} \quad   & \sigma_p^2 = \sum_{i=1}^{n}\sum_{j=1}^{n} e^{x_i} e^{x_j}\sigma_i \sigma_j \rho_{ij} \quad \text{(Portfolio risk)} \\
        \text{Subject to} \quad & \sum_{i=1}^{n} e^{x_i} = 1 \quad \text{(Sum of weights equals 1)}    
    \end{aligned}
\end{equation}


Finally we can use the softmax function to ensure that the sum of the weights equals 1, such as:

\begin{equation}
    w_i = \frac{e^{x_i}}{\sum_{j=1}^{n} e^{x_j}}
\end{equation}

Leading to the following problem formulation (with softmax):

\begin{equation}
    \begin{aligned}
        \text{Minimize} \quad   & \sigma_p^2 = \frac{1}{(\sum_{k=1}^{n} e^{x_k})^{2}} \sum_{i=1}^{n}\sum_{j=1}^{n} e^{x_i} e^{x_j}\sigma_i \sigma_j \rho_{ij} \quad \text{(Portfolio risk)} \\
    \end{aligned}
\end{equation}

\subsection*{Derivate of the objective function}

Now, let's delve into the derivate of the objective function:

We are going to derive it term by term using the chain rule, we first derive the term outside the sum such as :

\begin{equation}
    \frac{\partial}{\partial x_n} \frac{1}{(\sum_{k=1}^{N} e^{x_k})^{2}} = \frac{-2 e^{x_n}}{(\sum_{k=1}^{N} e^{x_k})^{3}}
\end{equation}

Where $N$ is the number of assets in the portfolio and $x_n$ is the variable we are deriving with respect to.

The second term is a bit more complicated, we will use the product rule:

\begin{equation}
    \frac{\partial}{\partial x_n} \sum_{i=1}^{N}\sum_{j=1}^{N} e^{x_i} e^{x_j}\sigma_i \sigma_j \rho_{ij} = \sum_{i=1}^{N}\sum_{j=1}^{N}( \frac{\partial}{\partial x_n} e^{x_i} ) e^{x_j}\sigma_i \sigma_j \rho_{ij} + \sum_{i=1}^{N}\sum_{j=1}^{N} e^{x_i}( \frac{\partial}{\partial x_n} e^{x_j} )\sigma_i \sigma_j \rho_{ij} 
\end{equation}

Because the two terms are similar, we will only focus on the first one:
We can see that the derivate is not null if $i \neq n$:

\begin{equation}
    \sum_{i=1}^{N}\sum_{j=1}^{N}( \frac{\partial}{\partial x_n} e^{x_i} ) e^{x_j}\sigma_i \sigma_j \rho_{ij} = \sum_{j=1}^{N} e^{x_n} e^{x_j}\sigma_n \sigma_j \rho_{nj} = \sigma_n e^{x_n} \sum_{j=1}^{N} e^{x_j}\sigma_j \rho_{nj}
\end{equation}

We can simplify the two sums by using the fact that $\rho_{ij} = \rho_{ji}$ and $\sigma_i \sigma_j = \sigma_j \sigma_i$:

\begin{equation}
    \frac{\partial}{\partial x_n} \sum_{i=1}^{N}\sum_{j=1}^{N} e^{x_i} e^{x_j}\sigma_i \sigma_j \rho_{ij} = 2 \sigma_n e^{x_n} \sum_{j=1}^{N} e^{x_j}\sigma_j \rho_{nj}
\end{equation}

Finally, we can derive the whole objective function:

\begin{equation}
    \begin{aligned}
        \frac{\partial}{\partial x_n} \sigma_p^2 & = \frac{-2 e^{x_n}}{(\sum_{k=1}^{N} e^{x_k})^{3}} \sum_{i=1}^{N}\sum_{j=1}^{N} e^{x_i} e^{x_j}\sigma_i \sigma_j \rho_{ij} + 2 \sigma_n e^{x_n} \sum_{j=1}^{N} e^{x_j}\sigma_j \rho_{nj} \\
    \end{aligned}
\end{equation}


\end{document}


