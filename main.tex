% create a basic latex article

\documentclass[12pt]{article}

\usepackage{amsmath}
\usepackage{amssymb}
\usepackage{graphicx}

\title{Seminar - Markowitz Portfolio Optimization}
\author{PUJOL Martin \\ RAMPONT Martin \\ THOMASSIN Pablo \\ STRIEBIG Maximilien}

\date{\today}

\documentclass{article}

\usepackage{amsmath}

\begin{document}

\maketitle

\section*{Theoretical Part}

\subsection*{Problem Description}

In the realm of financial portfolio management, the Markowitz portfolio optimization problem is a classical and essential topic. The primary objective is to allocate weights to different assets in a portfolio to maximize the expected return while minimizing the overall portfolio risk. Let's consider a portfolio with $n$ assets. The goal is to find the optimal set of weights for these assets.

\subsubsection*{Formalization}

Let:
\begin{align*}
    r_i      & : \text{Expected return of asset } i                 \\
    \sigma_i & : \text{Volatility (risk) of asset } i               \\
    w_i      & : \text{Weight of asset } i \text{ in the portfolio} \\
\end{align*}

The objective is to find the vector of weights $\mathbf{w} = [w_1, w_2, \ldots, w_n]$ that maximizes the expected portfolio return $\mu$ while minimizing the portfolio risk $\sigma_p$:

\begin{equation}
    \begin{aligned}
        \text{Maximize} \quad   & \mu = \sum_{i=1}^{n} r_i w_i                                                                                     \\
        \text{Subject to} \quad & \sigma_p = \sqrt{\sum_{i=1}^{n}\sum_{j=1}^{n} w_i w_j \sigma_i \sigma_j \rho_{ij}} \quad \text{(Portfolio risk)} \\
                                & \sum_{i=1}^{n} w_i = 1 \quad \text{(Sum of weights equals 1)}                                                    \\
                                & w_i \geq 0 \quad \text{(Non-negativity constraint)}
    \end{aligned}
\end{equation}

Where:
\begin{align*}
    \rho_{ij} & : \text{Correlation coefficient between assets } i \text{ and } j
\end{align*}

\section*{Numerical Part}

\subsection*{Selected Optimization Methods}

For solving the Markowitz portfolio optimization problem, we have chosen two numerical optimization methods:

\begin{enumerate}
    \item Method 1: [Insert Method 1 Name]
    \item Method 2: [Insert Method 2 Name]
\end{enumerate}

\subsection*{Algorithm Implementation}

Below are the basic functions describing the two chosen algorithms:

\subsubsection*{Method 1: [Insert Method 1 Name]}

[Insert code or pseudocode for Method 1 implementation]

\subsubsection*{Method 2: [Insert Method 2 Name]}

[Insert code or pseudocode for Method 2 implementation]

\subsection*{Results and Analysis}

We have applied both methods to the Markowitz portfolio optimization problem and obtained the following results:

[Insert results, tables, or graphs]

\subsubsection*{Interpretation}

[Provide interpretation of the results]

\subsubsection*{Comparison}

To compare the two methods, we analyze factors such as computational time and the number of iterations:

[Insert comparison results]

\end{document}


